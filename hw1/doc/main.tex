\documentclass[a4paper]{article}


\usepackage{geometry}
\usepackage{bm}
\usepackage{cmap}
\usepackage{ctex}
\usepackage{cite}
\usepackage{color}
\usepackage{float}
\usepackage{xeCJK}
\usepackage{amsthm}
\usepackage{amsmath}
\usepackage{amssymb}
\usepackage{setspace}

\usepackage{enumerate}
\usepackage{indentfirst}
\usepackage{graphicx}
\usepackage[cache=false]{minted}

\begin{document}

\section{思考题1:超前进位加法器}
	串行的加法器需要一位一位运算,时间是累加的。而超前进位加法器是并行计算,同时计算4位的所有值,而每个数位的计算都是由主析取范式构成,只需并联这些电路,每段电路内部串联即可,变得短了很多,运行速度就变快了。
\section{思考题2:wire和reg}
	wire 需要持续的驱动,只能用于组合逻辑。
	用途:
	\begin{enumerate}
		\item module的input和output端口
		\item 连接module里的不同部分
	\end{enumerate}

	reg 保存最后一次的赋值,既可用于组合逻辑,也可用于时序逻辑。
\section{拓展题}
	当reg赋值的触发条件是\textbf{赋值语句右侧任意操作数变化},则可以变成wire。

	当reg赋值的触发条件是\textbf{时钟的上升沿或下降沿}(触发器) 或\textbf{某一信号的高电平或低电平}(锁存器)则不能变成wire
\end{document}
